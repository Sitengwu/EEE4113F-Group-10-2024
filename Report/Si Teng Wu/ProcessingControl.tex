\documentclass[class=report,11pt,crop=false]{standalone}
% \usepackage{array}
\usepackage{adjustbox}
% Page geometry
\usepackage[a4paper,margin=20mm,top=25mm,bottom=25mm]{geometry}

% Font choice
\usepackage{lmodern}

\usepackage{lipsum}

% Use IEEE bibliography style
\bibliographystyle{IEEEtran}

% Line spacing
\usepackage{setspace}
\setstretch{1.20}

% Ensure UTF8 encoding
\usepackage[utf8]{inputenc}

% Language standard (not too important)
\usepackage[english]{babel}

% Skip a line in between paragraphs
\usepackage{parskip}

% For the creation of dummy text
\usepackage{blindtext}

% Math
\usepackage{amsmath}

% Header & Footer stuff
\usepackage{fancyhdr}
\pagestyle{fancy}
\fancyhead{}
\fancyhead[R]{\nouppercase{\rightmark}}
\fancyfoot{}
\fancyfoot[C]{\thepage}
\renewcommand{\headrulewidth}{0.0pt}
\renewcommand{\footrulewidth}{0.0pt}
\setlength{\headheight}{13.6pt}

% Epigraphs
\usepackage{epigraph}
\setlength\epigraphrule{0pt}
\setlength{\epigraphwidth}{0.65\textwidth}

% Colour
\usepackage{color}
\usepackage[usenames,dvipsnames]{xcolor}

% Hyperlinks & References
\usepackage{hyperref}
\definecolor{linkColour}{RGB}{77,71,179}
\hypersetup{
    colorlinks=true,
    linkcolor=linkColour,
    filecolor=linkColour,
    urlcolor=linkColour,
    citecolor=linkColour,
}
\urlstyle{same}

% Automatically correct front-side quotes
\usepackage[autostyle=false, style=ukenglish]{csquotes}
\MakeOuterQuote{"}

% Graphics
\usepackage{graphicx}
\graphicspath{{Images/}{../Images/}}
\usepackage{makecell}
\usepackage{transparent}

% SI units
\usepackage{siunitx}

% Microtype goodness
\usepackage{microtype}

% Listings
\usepackage[T1]{fontenc}
\usepackage{listings}
\usepackage[scaled=0.8]{DejaVuSansMono}

% Custom colours for listings
\definecolor{backgroundColour}{RGB}{250,250,250}
\definecolor{commentColour}{RGB}{73, 175, 102}
\definecolor{identifierColour}{RGB}{196, 19, 66}
\definecolor{stringColour}{RGB}{252, 156, 30}
\definecolor{keywordColour}{RGB}{50, 38, 224}
\definecolor{lineNumbersColour}{RGB}{127,127,127}
\lstset{
  language=Matlab,
  captionpos=b,
  aboveskip=15pt,belowskip=10pt,
  backgroundcolor=\color{backgroundColour},
  basicstyle=\ttfamily,%\footnotesize,        % the size of the fonts that are used for the code
  breakatwhitespace=false,         % sets if automatic breaks should only happen at whitespace
  breaklines=true,                 % sets automatic line breaking
  postbreak=\mbox{\textcolor{red}{$\hookrightarrow$}\space},
  commentstyle=\color{commentColour},    % comment style
  identifierstyle=\color{identifierColour},
  stringstyle=\color{stringColour},
   keywordstyle=\color{keywordColour},       % keyword style
  %escapeinside={\%*}{*)},          % if you want to add LaTeX within your code
  extendedchars=true,              % lets you use non-ASCII characters; for 8-bits encodings only, does not work with UTF-8
  frame=single,	                   % adds a frame around the code
  keepspaces=true,                 % keeps spaces in text, useful for keeping indentation of code (possibly needs columns=flexible)
  morekeywords={*,...},            % if you want to add more keywords to the set
  numbers=left,                    % where to put the line-numbers; possible values are (none, left, right)
  numbersep=5pt,                   % how far the line-numbers are from the code
  numberstyle=\tiny\color{lineNumbersColour}, % the style that is used for the line-numbers
  rulecolor=\color{black},         % if not set, the frame-color may be changed on line-breaks within not-black text (e.g. comments (green here))
  showspaces=false,                % show spaces everywhere adding particular underscores; it overrides 'showstringspaces'
  showstringspaces=false,          % underline spaces within strings only
  showtabs=false,                  % show tabs within strings adding particular underscores
  stepnumber=1,                    % the step between two line-numbers. If it's 1, each line will be numbered
  tabsize=2,	                   % sets default tabsize to 2 spaces
  %title=\lstname                   % show the filename of files included with \lstinputlisting; also try caption instead of title
}

% Caption stuff
\usepackage[hypcap=true, justification=centering]{caption}
\usepackage{subcaption}

% Glossary package
% \usepackage[acronym]{glossaries}
\usepackage{glossaries-extra}
\setabbreviationstyle[acronym]{long-short}

% For Proofs & Theorems
\usepackage{amsthm}

% Maths symbols
\usepackage{amssymb}
\usepackage{mathrsfs}
\usepackage{mathtools}

% For algorithms
\usepackage[]{algorithm2e}

% Spacing stuff
\setlength{\abovecaptionskip}{5pt plus 3pt minus 2pt}
\setlength{\belowcaptionskip}{5pt plus 3pt minus 2pt}
\setlength{\textfloatsep}{10pt plus 3pt minus 2pt}
\setlength{\intextsep}{15pt plus 3pt minus 2pt}

% For aligning footnotes at bottom of page, instead of hugging text
\usepackage[bottom]{footmisc}

% Add LoF, Bib, etc. to ToC
\usepackage[nottoc]{tocbibind}

% SI
\usepackage{siunitx}

% For removing some whitespace in Chapter headings etc
\usepackage{etoolbox}
\makeatletter
\patchcmd{\@makechapterhead}{\vspace*{50\p@}}{\vspace*{-10pt}}{}{}%
\patchcmd{\@makeschapterhead}{\vspace*{50\p@}}{\vspace*{-10pt}}{}{}%
\makeatother
\makenoidxglossaries

\newacronym{radar}{RADAR}{Radio Detection and Ranging}
\begin{document}
% ----------------------------------------------------
\chapter{Processing and Control}

\textbf{\textit{Prepared by Si Teng Wu - WXXSIT001}}
\vspace{0.5cm}
% ----------------------------------------------------
\section{Introduction}
This subsystem involves designing a hardware solution and software for controlling the collection, processing, storage and transmission of data as well as an identification solution. It has been divivded into the following submodules: Identification, Weight Processing, Storage, Transmission, System Control and Power Saving Techniques. Throguh the use of a micro-controller module, and various other hardware modules, this subsytem will be the "brain" of the system and act as the central hub for data while the system is in the field. This subsystem will receive input data from the Scale and Power susbsystems and provide output data to the UI subsystem.

\section{Requirements and Specifications}
\subsection*{User Requirements}
The stakeholder/user, Carrie Hickman, a PhD student at the Fitzpatrick Institute of African Ornithology, UCT, has provided the following user requirements that are releveant to this susbsystem for her problem. %These requirements were collected through stakeholder engagements of the EEE4113F Teams \cite{stakeholder}.
\begin{itemize}
    \item The bird being wieghed needs to be identified.
    \item The weight should be accurate to within a gram.
    \item There should be some way to collect data without climbing the tree.
    \item The user specified data should be stored on an SD card as back-up.
    \item The system should last two weeks.
\end{itemize}


\subsection*{Requirements Analysis}

Using these user requirements, the following functional requirements (FR), specfications (SP) and Acceptance Test Procedures (ATP) were created. The specfications provides a standard for the subsystem and the ATPs will be used to test the standards of the proposed solution.

\begin{table}[h]
    \adjustbox{width=\textwidth}
    {\begin{tabular}{|c|p{0.8\linewidth}|}
            \hline
            \textbf{FR ID} & \textbf{Functional Requirement}                                                \\
            \hline
            FR-1 & Control the system using a microcontroller. \\
            \hline
            FR-2 & Use a suitable identification technique on the bird as it is being weighed. \\
            \hline
            FR-3 & Collect diagnostic battery data. \\
            \hline
            FR-4 & Collect the weight data. \\
            \hline
            FR-5 & Develop a weight processing algorithm that can process the weight data. \\
            \hline
            FR-6 & Store the data on local, non-volatile and removable storage. \\
            \hline
            FR-7 & Transmit the data wirelessly from the nest to the user's device on the ground. \\
            \hline
            FR-8 & Use power saving techniques to limit the use of power. \\
            \hline
        \end{tabular}}
    \caption{Functional Requirements of the subsystem}
    \label{tab:FR}
\end{table}

\begin{table}[h]
    \centering
    \small
    \adjustbox{width=\textwidth}
    {\begin{tabular}{|c|p{0.6\linewidth}|c|c|}
    \hline
    \textbf{SP ID} & \textbf{Specification} & \textbf{FR ID} & \textbf{ATP ID} \\
    \hline
    SP-1 & Use the ESP32 S3 Dev Module operating at 3.3V and <250mA. & FR-1 & ATP-1 \\
    \hline
    SP-2 & Use the RDM6300 125kHz RFID module operating at 5V <50mA. & FR-2 & ATP-2 \\
    \hline
    SP-3 & The RFID must detect PIT tags at a minimum distance of 5cm and use GPIO pins to switch on/off.  & FR-2 & ATP-3 \\
    \hline
    SP-4 & Activate RFID only when the weight exceeds 2.2 kg. & FR-2, FR-8 & ATP-4 \\
    \hline
    SP-5 & Use the ADC to read the battery level and calculate it as a percentage. & FR-3 & ATP-5 \\
    \hline
    SP-6 & Read the HX711 using I2C every 10 seconds. & FR-4, FR-8 & ATP-6 \\
    \hline
    SP-7 & Implement a suitable filtering technique (normal average, moving average, exponential average, or median filter) to process the weight data. & FR-4 & ATP-7 \\
    \hline
    SP-8 & Utilize a micro SD card module connected via SPI to store data in a text file format "<weight,ID,battery\%>". & FR-6 & ATP-8 \\
    \hline
    SP-9 & Implement a web server over WiFi to transmit data. & FR-7 & ATP-9 \\
    \hline
    SP-10 & Utilize deep sleep mode whenever possible to minimize power consumption. & FR-8 & ATP-10 \\
    \hline
    SP-11 & Use a 433MHz remote transmitter and receiver to start the WiFi when necessary. & FR-7, FR-8 & ATP-11 \\
    \hline
\end{tabular}}
\caption{System Specifications and associated FRs and ATPs}
\label{tab:SP}
\end{table}
    
\section{Acceptance Test Procedures}

\section{Design Choices}

\section{Results}

\section{ATP Results}

% ----------------------------------------------------
\ifstandalone
    \bibliography{../Bibliography/References.bib}
    \printnoidxglossary[type=\acronymtype,nonumberlist]
\fi
\end{document}
% ----------------------------------------------------